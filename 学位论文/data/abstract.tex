% !TeX root = ../thuthesis-example.tex

% 中英文摘要和关键字

\begin{abstract}\label{abstract}

随着编程领域的快速发展,程序员为了更好的学习与交流,会在程序员问答网站提出、讨论问题,并且附上代码片段。在讨论过程中,由于代码片段往往是用于描述问题或者解决方案的关键逻辑,其可能不是完整可编译执行的,还可能存在安全性和效率等问题。程序开发人员在借鉴问答网站上的知识时,他们会直接地,或在经过局部地改写后复制代码片段到自己的软件系统中。在这个过程中,一段可用的、并未验证安全性与高效性的代码片段有可能通过程序员问答网站传播到应用程序中,从而影响应用程序的质量。

大部分已有的程序员问答网站相关研究集中在C/C++、Java语言相关的帖子,高度依赖语言独有的基础工具。随着区块链技术的快速发展,智能合约作为运行在区块链上的程序,程序员也对其开发中的问题展开了广泛的讨论,积累了大量的帖子。然而目前还没有大规模的对程序员问答网站下智能合约的实证研究。由于问答中的代码片段大多为不完整的、无法编译的代码段,为了对这些代码中存在的安全性、高效性的问题开展实证研究,本文提出了基于抽象语法树的代码克隆检测方法。该方法提取不完整的,无法编译的智能合约代码片段的抽象语法树,在保证语法一致的情况下,将抽象语法树与开源项目的抽象语法树进行比较,寻找语义相似的代码克隆对。实验表明,基于抽象语法树的代码克隆方法能以较高的准确率找到正确的代码克隆对。基于该方法,本文在程序员问答网站上1120个智能合约相关的讨论帖与609个开源智能合约中,面向智能合约常见漏洞与Gas低效模式进行了实证研究。本文在开源项目上预先进行漏洞与Gas低效模式检测,然后根据代码克隆方法将开源项目的漏洞与Gas低效模式定位到问答网站上的代码片段;并且研究在程序员问答网站上漏洞与Gas低效模式的分布与可能的出现原因,检测所找出漏洞与低效模式在现实环境的扩散情况。实验表明,代码克隆方法保证了被映射到代码片段的漏洞与Gas低效模式与开源项目的漏洞与Gas低效模式的一致性;程序员问答网站上包含一定量的漏洞与Gas低效模式,并且在现实存在的智能合约项目中存在。

  % 关键词用“英文逗号”分隔,输出时会自动处理为正确的分隔符
  \sysusetup{
    keywords = {问答网站, 代码克隆, 漏洞检测, Gas},
  }
\end{abstract}

\begin{abstract*}


With the rapid development of the programming field, programmers ask and discuss problems and attach code snippets on programmer Q\&A sites for better learning and communication. During the discussion, since code snippets are often used to describe the key logic of a problem or solution, they may not be fully compilable and executable, and may also have issues such as security and efficiency. When program developers borrow knowledge from Q\&A sites, they copy code snippets directly, or after partial rewriting, into their own software systems. In the process, a usable code fragment that has not been validated for security and efficiency may be propagated to the application through the Q\&A site, thus affecting the quality of the application.

Most of the existing research of programmer Q\&A sites focus on posts related to C/C++ and Java languages, relying heavily on the language's unique underlying tools. With the rapid development of blockchain technology, smart contracts, as programs running on blockchain, have also been widely discussed by programmers on issues in their development, and a large number of posts have been accumulated. However, there is no large-scale empirical study of smart contracts under programmer Q\&A sites. Since most of the code fragments in Q\&A sites are incomplete and uncompilable code segments, in order to carry out empirical studies on the security and efficiency problems in these codes, this paper proposes a code clone detection method based on abstract syntax trees. The method extracts the abstract syntax tree of incomplete, uncompilable smart contract code fragments, and compares the abstract syntax tree with that of open source projects to find semantically similar code clone pairs, while ensuring syntactic consistency. The experiments show that the code cloning method based on the abstract syntax tree can find the correct code clone pairs with high accuracy. Based on this method, this paper conducts an empirical study in 1120 smart contract-related discussion posts and 609 open source smart contracts on the Programmer's Q\&A website, which is oriented towards common vulnerabilities and Gas inefficient patterns in smart contracts. The vulnerabilities and Gas inefficient patterns are pre-detected on the open source projects, and then the vulnerabilities and Gas inefficient patterns of the open source projects are located on the code snippets on the Q\&A website according to the code cloning method; the distribution and possible causes of the vulnerabilities and Gas inefficient patterns on the Q\&A website are also investigated, and the proliferation of the identified vulnerabilities and Gas inefficient patterns in the real environment is detected. The experiments demonstrate that the code cloning approach ensures the consistency of the vulnerabilities and Gas inefficient patterns mapped to code snippets with those of open source projects; a certain amount of vulnerabilities and Gas inefficient patterns exist on the Programmer's Q\&A site and in real-world smart contract projects.
 
  % Use comma as seperator when inputting
  \sysusetup{
    keywords* = {smart contract, Q\&A sites, code clone, bug detection,Gas},
  }
\end{abstract*}
