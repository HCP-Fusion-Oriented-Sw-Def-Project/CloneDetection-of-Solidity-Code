% !TeX root = ../thuthesis-example.tex

\chapter{总结与展望}\label{figures_tables}

%研究客体
问答网站是程序员的重要平台,因为它们提供了编程问题的问题描述与对应解答;在解答的文本之外,也提供了大量的代码示例帮助理解。这些文本与代码示例也能够更好的帮助研究人员理解程序员的关注重点。但这些有帮助的回答也有可能带来糟糕的结果;作为一名问题浏览者或者提出者,在看到了适用于自己项目,稍作修改就能直接运行的代码时,选择复制粘贴到自己的项目是可能的。然而问题解答者或者其他的编写了这段被复制代码片段的人不会考虑这种情况,他们往往专注于当前问题,仅提供最核心的代码以求清晰的回答问题。这产生了本文的研究问题:是否在问答网站的智能合约代码片段中存在危害如漏洞或者Gas低效模式?它们会在现实的开源合约中扩散吗?如何利用问答网站的数据,了解编程人员的热点问题,帮助问答网站增强答疑能力,减少问答网站糟糕的回答或者糟糕的代码,也会是问答网站研究领域下依旧存在的研究课题。

%总结每章内容
本文的前四章对提出的基于抽象语法树的代码克隆方法及其在问答网站上的实证研究进行了背景、模型和实验结果等方面的介绍,本章主要对论文的研究内容与成果进行总结,并分析目前工作的缺点以及改进方向,并对以后的研究进行了展望。

\section{工作总结}

%背景与研究方向
程序员问答网站是以帖子为提问单元,使用文本、图片与代码片段进行交流的。实例研究证明,程序员问答网站中的Android语言下代码片段存在安全漏洞,并且扩散于开源项目。在StackOverflow中的智能合约板块存在大量的代码片段,基于区块链的智能合约在安全性上有着不同的要求;同时,区块链的特性使得智能合约的运行消耗虚拟货币,高效的智能合约能够减少虚拟货币的损失。如果能够检验程序员问答网站中的代码片段的安全性与高效性,研究它们在开源项目中的扩散情况,将有助于增加程序员编写更安全与高效的智能合约。本文提出的相似匹配方法通过提取代码片段与开源项目的抽象语法树,并且根据语法一致、语义相似的逻辑获取代码克隆对。本文使用仅能用于开源项目的漏洞检测工具与Gas低效模式检测工具检测开源项目。代码克隆对保证了漏洞与Gas低效模式的一致性,存在于开源项目的危害能够定位到代码片段的相同位置。相似匹配方法帮助本文找到程序员问答网站中存在漏洞或Gas低效模式的代码片段。另外,本文研究了存在这两种危害的代码片段在开源项目中的扩散情况。下面是本文研究的主要内容:

\begin{itemize}
    \item 本文对国内外的问答网站相关论文进行调查分析,发现问答网站在智能合约相关分析的缺失。因此本文使用代码克隆的方法,借助智能合约相关研究进行工作。
    
    \item 本文针对开源项目与StackOverflow代码片段的区别,将它们共同存在的合约与函数提取作为实验的基本单元。以此为基础,本文对数据进行去除注释、规范格式的预处理。
    
    \item 本文设计并实现了一种代码克隆方法,通过将代码解析为抽象语法树,并且序列化其中的语法语义序列;它能够在合约与函数层面找到语法一致、语义相似的代码克隆对,实验证明,在抽样结果中,它在合约层面以90\%的比例与函数层面91\%的比例保证漏洞与Gas低效模式在克隆对的一致性。
    
    \item 本文使用基于抽象语法树的代码克隆方法,对StackOverflow的代码片段的漏洞与Gas低效模式进行的大范围的检测与分析,结果表明StackOverflow智能合约代码片段中的11.91\%包含漏洞、StackOverflow 智能合约的25.56\%的代码片段包含至少一个Gas低效模式。
    
    \item 文本对存在于StackOverflow代码片段的漏洞与低效模式在开源项目的可能扩散进行分析与研究,实验表明开源项目中,25.2\% 至少包含一个StackOverflow的代码片段,6.7\%包含至少一个不安全的代码片段,16.1\% 包含至少一个Gas低效的代码片段。
\end{itemize}
 

\section{未来展望}

本文提出一种基于抽象语法树的代码克隆检测方法,并且将其运用在StackOverflow的漏洞与Gas低效模式检测上,对于这些存在于StackOverflow的危害是否同样存在于开源项目中,本文通过对开源项目进行重用分析,确定了危害的扩散情况。

在本节中介绍未来工作的改进方向。

\begin{itemize}

    \item 本文的数据集数量仍然不足,虽然本文爬取了 StackOverflow 上所有关于solidity 的帖子,收集了 176,234 个开源项目,时间范围从 2018 年到 2020年。然而,开源智能合约只占以太坊上所有智能合约的一小部分。为了进一步改进实验,将在未来持续收集数据。
    
    \item 目前方法在合约与函数层面实现了代码克隆,粒度较粗,后续可在语句层面寻找漏洞与Gas低效模式定位一致的代码克隆。

    \item 工作目前仅完成实证研究,后续考虑为StackOverflow的用户进行知识推荐,针对StackOverflow上存在漏洞与Gas低效模式的代码片段,提供正确与高效的替换片段以及学习范例。
    
    %\item 在验证方法有效性的实验中,本文并没有构建一个漏洞定位一致但语法不一定一致的数据集用于验证召回率,本文尝试使用BLEU作为筛选后手动寻找的方法,但它的效率非常低。后续本文将寻找更好的构建数据集的方法验证本文的方法有效性。

\end{itemize}




%\cleardoublepage
